%&LaTeX
\batchmode
\documentclass{article}
\usepackage{graphicx}
\usepackage{hyperref}

\newcommand{\tab}{\hspace{5mm}}

\begin{document}

\begin{center}
\textbf{{\huge Glossary for STIR}}\\
\textbf{C. Labb\'{e}, H. Zaidi, C. Morel, K. Thielemans, C. Tsoumpas}
\end{center}

\begin{center}
\textit{Version 3.0}\\
Originally based on PARAPET Deliverable 4.1,
Extended for Quantitative Reconstruction and motion compensation

\end{center}

\section*{Introduction}
The purpose of this document is to define the terms used by STIR, as not everyone in the nuclear 
medicine community uses the same words. In addition, it also gives a brief overview and explanation 
of data concepts. However, this is all kept brief. Please read additional material, for instance
[Fah2002].

\section*{Basics}

\begin{description}
% ROW 2
\item[Geometry] 
A cylindrical geometry has been chosen to describe positron 
tomographs made of a number of adjacent detector rings and reconstructed 
image volumes. The geometry supports consequently two principal 
directions \textit{axially} along the scanner cylinder and \textit{transaxially} 
perpendicular to the cylinder axis.
% ROW 3
\item[Scanner] 
Geometrically associated to the cylindrical volume defined by 
the inner dimensions of the positron tomograph.
% ROW 4
\item[Detector ring ] 
Geometrically associated to the cylindrical volume defined by 
the dimensions of a detector ring. Note that because currently 
Depth of Interaction is not taken into account, the effective ring radius 
used in the building blocks is the sum of the inner ring radius 
and an average depth of interaction (e.g. \ensuremath{\sim} 1cm for BGO).
% ROW 5
\item[Detector] 
Sometimes called \textbf{detector crystal}. Geometrically associated 
to the inner face of a detector element. The \textbf{scanner} is then 
considered as a tessellation of \textbf{detectors} constructing adjacent \textbf{rings}. 
For many scanners, detectors are organized in a block. For instance, 
on the HR+ scanner, a detector block consists of 8x8 detectors.
% ROW 6
\item[LOR (Line-Of-Response)] 
Line joining the centres of two \textbf{detectors}. Ignoring scatter, 
attenuation and other physical effects, the average number of 
coincidences observed between two detectors can be estimated 
as the line integral of the tracer distribution along the \textbf{LOR}. 
This does not take the finite width of the \textbf{TOR} into account, 
nor scatter within the detectors. It can be shown that this line 
integral approximation works best for LORs that do not run parallel 
to edges within the object. We say that the projector that uses 
this model is a \textbf{ray tracing} projector.
% ROW 7
\item[TOR (Tube of response)] 
Tube joining two \textbf{detectors}. 
% ROW 8
\item[Sinogram] 
Set of \textbf{bins} corresponding to 1 segment and 1 \textbf{axial} position. 
Before \textbf{axial compression,} this corresponds to LORs in a \textbf{detector} \textbf{ring} 
(\textit{direct} \textit{sinogram}) or between two different \textbf{detector} \textbf{rings} 
(\textit{oblique} \textit{sinogram}). For a \textbf{scanner} of $n$ \textbf{detector 
rings}, there are $n$ \textbf{direct sinograms} and $n^2-1$ \textbf{oblique 
sinograms} for a total of $n^2$ \textbf{sinograms}. With \textbf{axial 
compression}, the number of \textbf{direct sinograms} is $2n-1$. Conventionally, 
the \textbf{view} angle in an oblique sinogram runs only over 180 
degrees, meaning that only half of the detectors in each ring 
are covered. The other half corresponds to the \textbf{sinogram} in 
the opposite \textbf{segment} (with minus the average ring difference),
% ROW 9
\item[View] 
The azimuthal angle of an \textbf{LOR} (ignoring \textbf{interleaving}, 
see the documentation of the ProjDataInfoCylindricalNoArcCorr 
class, and \textbf{mashing}).
% ROW 10
\item[Bin] 
A single element in a sinogram, completely specified by its \textbf{segment, 
axial} \textbf{position, view} and \textbf{tangential position}.
% ROW 11
\item[Ring difference] 
Number of \textbf{rings} between two \textbf{rings} associated to a \textbf{sinogram}. 
If \textit{ringA} and \textit{ringB} are the ring numbers, the \textbf{ring 
difference} is given by \textit{ringB} -- \textit{ringA}. Thus there can be \textit{positive} 
and \textit{negative} \textbf{ring differences}. \linebreak
The (average) \textbf{ring difference} of a \textbf{direct sinogram} is 
zero.
% ROW 12
\item[Michelogram] 
Representation of \textbf{sinograms} on a square grid as shown in 
Annex 1. If \textit{ringA} and \textit{ringB} are the ring numbers associated 
to a \textbf{sinogram}, \textit{ringA} is represented on the horizontal 
axis and \textit{ringB} on the vertical axis. \textbf{Positive ring differences} 
are below the line representing \textbf{direct sinograms} and \textbf{negative 
ring differences} above this line.
% ROW 13
\item[Segment] 
Set of \textbf{merged} \textbf{sinograms} with a common average \textbf{ring 
difference} as shown in Annex 1.
% ROW 14
\item[Viewgram] 
Set of equal azimuth \textbf{merged} \textbf{LORs} of a \textbf{segment}.
% ROW 15
\item[Projection data]
The set of all (measured) LORs, normally split into \textbf{segments} etc.
The word ``projection'' is used because after various corrections and
ignoring noise, the measured data can be approximated as line integrals 
through the object.
\item[FOV (Field-Of-View)] 
Geometrically associated to the volume for which there is at 
least 1 \textbf{bin} with non-zero detection probability\textbf{.} In many 
cases, the term is also used for the smaller volume for which 
there is at least 1 \textbf{bin} with non-zero detection probability 
for every \textbf{view}. The latter FOV is usually cylindrical.
% ROW 16
\item[Image slice] 
Geometrically associated to a cylindrical volume defined by 
a slice of the \textbf{FOV}. By convention, a \textbf{slice} is half the 
width of a \textbf{ring}. For a \textbf{scanner} of \textit{n} \textbf{detector} \textbf{rings}, 
there are 2\textit{n}--1 \textbf{image slices}.
% ROW 17
\item[Direct plane] 
\textbf{Image slice} centered on a \textbf{ring}. For a \textbf{scanner} of \textit{n} \textbf{detector} \textbf{rings}, 
there are \textit{n} \textbf{direct planes}. The \textbf{FOV} is ended by two \textbf{direct 
planes} centered on the first and last \textbf{rings}.
% ROW 18
\item[Cross plane] 
\textbf{Image slice} in between two consecutive \textbf{direct planes}. \textbf{Direct 
planes} are adjacent to \textbf{cross planes}. For a \textbf{scanner} of \textit{n} \textbf{detector} \textbf{rings}, 
there are 2\textit{n}--1 \textbf{cross planes}.
\end{description}

\section*{Different data compressions used in PET data}
\begin{description}
% ROW 21
\item[Trimming] 
Reduction of the number of \textbf{bins} in tangential direction without 
changing the size of \textbf{bins}. \textbf{Trimming} is a type of \textbf{bin} 
truncation.
% ROW 22
\item[Angular compression (Mashing)] 
Reduction of the number of \textbf{views} by a multiple of two. As 
an example, doing a \textbf{mashing} of 2 means that pairs of \textbf{views} 
have been added 2 by 2 to form only one \textbf{view}.
% ROW 23
\item[Axial compression (Span)] 
Reduction of the number of \textbf{sinograms} at different \textbf{ring 
differences} as shown in Annex 1. \textbf{Span} is a number used by 
Siemens/CTI to say how much axial compression has been used. It is always 
an odd number. Higher span, more axial compression. Span 1 means 
no axial compression. Note that GE scanners uses mixed data, 
where segment 0 has span 3, while other segments have span 1 (or 2, depending on how you look at it, 
see the Annex).
\end{description}

\section*{Terms used in quantitative PET reconstruction}
\begin{description}
% ROW 24
\item[Scatter Point]
Coordinate where a scatter event takes place. 
% ROW 25
\item[SSS - Single scatter simulation]
Estimation of the probability to measure a coincidence event that one of
the two photons has been scattered only once. 
% ROW 26
\item[B-Splines]
Basis splines are a set of polynomial functions that have minimal support with 
respect to a given degree, smoothness, and domain partition. In imaging they 
are useful for performing very fast multidimensional interpolation calculations.
% ROW 27
\item[Inverse-SSRB]
It is the pseudo-inverse operation of single slice rebinning which can be used 
as the simplest way to extrapolate direct sinograms into indirect sinograms. 
% ROW 28
\item[Plasma Data]
Radioactivity concentration in plasma (and blood) during the scanning acquisition.
Usually it is measured in $\mathit{kBq/cm^3}$ over a time window of 1 second. 
% ROW 29
\item[Dynamic Data/Images]
A stack of projection data or images through time.
% ROW 30
\item[Kinetic Model]
The kinetic model describe the tracer exchange between plasma and tissue 
and between tissue compartments.
% ROW 31
\item[Kinetic Parameters]
The parameters of the kinetic model which are estimated such that 
the model is in agreement with the acquired data. 
% ROW 32
\item[(Kinetic) Model Matrix]
Linear kinetic models can be written with compact matrix operations, which 
relate the dynamic images and the kinetic parameters with the kinetic model matrix. 
This matrix can be seen as the application of the transformation from parametric 
domain to the temporal domain. 
% ROW 33
\item[Patlak Plot]
For irreversible tracers, after a certain period from tracer injection, 
the free tracer in tissue reaches equilibrium with the radiotracer in plasma and 
then the original model simplifies to a linear plot known as the Patlak Plot.
% ROW 34
\item[Parametric Image]
An image whose voxels hold the values the kinetic parameters.
% ROW 35
\item[Parametric Image Reconstruction (PIR)]
Estimation of the kinetic parameters from dynamic images for each voxel (indirect PIR). 
The parametric images can also be reconstructed directly from dynamic projection data. 
\end{description}

\section*{Terms used in motion-compensated reconstruction}

\begin{description}
\item[Gated Data/Images]
PET acquisitions can be gated according to an external signal (e.g. respiration, ECG). Gated projection data or gated images correspond to one cardiac/respiratory position of the cardiac/respiratory cycle. 


\item[Motion Vectors]
3D vectors that store the information of the location where the activity was originally. Information is stored in x-y-z motion vectors and they relate the image at the reference position with the image at the corresponding gate.


\item[Motion Compensated Image Reconstruction (MCIR)]
Incorporates motion information within reconstruction so that a motion-free image is directly reconstructed. 

\item[Reconstruct-Transform-Average (RTA)]
Incorporates motion information after reconstruction to obtain a motion-free image.
\end{description}

\newpage
 \section*{ ANNEX 1 :Michelogram}

Michelogram of data for the ART scanner with a maximum \textbf{ring 
difference} of 17 and a \textbf{span} of 7. \\
Segment 0 is on the diagonal. \\
Each dot would correspond to a single sinogram if no axial compression 
would be used. Axial compression consists in adding sinograms 
together whose central LORs intersect the scanner axis 
in the same point. This can be seen as a generalization of the 
Single Slice Rebinning Algorithm. In the drawing below, the diagonal 
lines connecting the dots indicate the sinograms that are added 
together. The illustration is for \textbf{span} 7=4+3 (this terminology 
was introduced because for some axial positions, 4 sinograms 
are added, while for others only 3. Note that \textbf{span} is always 
odd.).\\
Note that STIR supports data with different \textbf{span} for each \textbf{segment}. 
In particular, STIR currently interpretes the (3D) data organisation of the
GE Advance and GE Discovery scanners as using segment 0 with span=3, while 
other segments are span=1. Another interpretation of GE data is that segment 0 uses
span=3, while other segments use span=2 (i.e. segment 1 contains ring differences +2 and +3), 
see the slide on p.7 of [Fay2004].

\begin{figure}[htbp]
\begin{center}
\includegraphics[width=6.in, height=6in]{graphics/STIR-glossaryFig1}
\caption{Michelogram with span=7. \textbf{Warning}: due to historical 
reasons, the axis labels are wrong. The horizontal axis corresponds to
\textit{ringB}.}
\end{center}
\end{figure}


Legend: 
\begin{description}
\item[rdmin] the minimum ring difference
\item[rdmax] the maximum ring difference
\item[average\_delta] the averaged ring difference. This is equal 
to segment\_num*span if the span is the same for each segment.
\end{description}

\section*{References}

[Fah2002] FH Fahey, "Data Acquisition in PET Imaging", J Nucl Med Technol 2002; 30:39, 49.

[Fah2004] FH Fahey, 2004 SNM Mid-Winter Educational Symposium slides, online at
\url{http://interactive.snm.org/docs/Fahey%20presentation.pdf}
\end{document}
